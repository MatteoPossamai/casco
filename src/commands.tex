								 
\newcommand{\hide}[1]{%
    \IfEqCase{\hideComments}{%
        {false}{#1}%
        {true}{}%
        % you can add more cases here as desired
    }
}

\newcommand{\techReportAppendix}[1]{%
    \IfEqCase{\shortVersion}{%
        {false}{Appendix \ref{#1}}%
        {true}{\cite{techReport}}%
        % you can add more cases here as desired
    }
}%

\newcommand{\techReportAppendices}[2]{%
    \IfEqCase{\shortVersion}{%
        {false}{Appendices \ref{#1}--\ref{#2}}%
        {true}{\cite{techReport}}%
        % you can add more cases here as desired
    }
}%


\newcommand{\onlyTechReport}[1]{%
    \IfEqCase{\shortVersion}{%
        {false}{#1}%
        % you can add more cases here as desired
    }
}%


\newcommand{\onlyShortVersion}[1]{%
    \IfEqCase{\shortVersion}{%
        {true}{#1}%
        % you can add more cases here as desired
    }
}%


% NAMES
\newcommand{\lang}{$\mu$\textsc{Asm}}
\newcommand{\tightpar}[1]{{\smallskip \noindent\bf #1}}
\newcommand{\para}[1]{\subsubsection*{#1}}
\newcommand{\spectre}{\textsc{Spectre}}
\newcommand{\tool}{\textsc{Spectector}\xspace}
% SYSTEMS
\newcommand{\clang}{\textsc{Clang}}
\newcommand{\icc}{\textsc{Icc}}
\newcommand{\vcc}{\textsc{Visual C++}}
\newcommand{\gcc}{\textsc{Gcc}}
\newcommand{\ciao}{\textsc{Ciao}}

%%%%% automatically break the line at commas in math mode
\AtBeginDocument{%
  \mathchardef\mathcomma\mathcode`\,
  \mathcode`\,="8000 
}
{\catcode`,=\active
  \gdef,{\mathcomma\discretionary{}{}{}}
}

\newcommand\fcirc{{\color{black}\bullet}\mathllap{\circ}}

%%%%%%%% ENVIRONMENTS
\theoremstyle{definition}
\newtheorem{definition}{Definition}
\newtheorem{exampleInt}{Example}
\theoremstyle{plain}
\newtheorem{theorem}{Theorem}
\newtheorem{corollary}{Corollary}
\newtheorem{lemma}{Lemma}
\newtheorem{proposition}{Proposition}
\newenvironment{example}
 {\begin{exampleInt} } 
 { $\hfill \blacksquare$\end{exampleInt} }
\newenvironment{proofsketch}{\paragraph*{Proof sketch}}{\\}


%%%%%%%% LANGUAGES

% Jose: Fix for \lstinline + ttfamily + math mode
% https://tex.stackexchange.com/questions/127010/how-can-i-make-lstinline-function-normally-inside-math-mode
\usepackage{letltxmacro}
\newcommand*{\SavedLstInline}{}
\LetLtxMacro\SavedLstInline\lstinline
\DeclareRobustCommand*{\lstinline}{%
  \ifmmode
    \let\SavedBGroup\bgroup
    \def\bgroup{%
      \let\bgroup\SavedBGroup
      \hbox\bgroup
    }%
  \fi
  \SavedLstInline
}

\lstset{numbers=left,xleftmargin=2em,frame=single,framexleftmargin=1.5em,basicstyle=\ttfamily}

\lstdefinestyle{Cstyle}
{
	%% FORMATTING
	frame = tb,
  belowskip=.4\baselineskip,
  aboveskip=.4\baselineskip,
  	showstringspaces = false,
  	breaklines = true,
  	breakatwhitespace = true,
  	tabsize = 3,
  	numbers = left,
    stepnumber = 1,
    numberstyle = \tiny\color{gray},
  	%%% LANGUAGE
    language = {[ANSI]C},
    alsoletter={.\$},
    basicstyle={\ttfamily\color{black}},
    stringstyle={\ttfamily\color{string-color}},
    keywordstyle={\ttfamily\color{Blue3}},
    keywordstyle=[2]{\ttfamily\color{Green4}},
    keywordstyle=[3]{\ttfamily\color{orange}},
    keywordstyle=[4]{\ttfamily\color{violet}},
    otherkeywords = {temp,y,A,B,k,size},
    morekeywords = [2]{A,B},
    morekeywords = [3]{},
    morekeywords = [4]{y,size,k,temp},
}

\lstdefinestyle{ASMstyle}
{
	%% FORMATTING
	frame = tb,
  belowskip=.4\baselineskip,
  aboveskip=.4\baselineskip,
  	showstringspaces = false,
              	breaklines = true,
  	breakatwhitespace = true,
  	tabsize = 3,
  	numbers = left,
    stepnumber = 1,
    numberstyle = \tiny\color{gray},
  	%%% LANGUAGE
    alsoletter={.\$\%},
    basicstyle={\ttfamily\color{black}},
    stringstyle={\ttfamily\color{string-color}},
    keywordstyle={\ttfamily\color{Blue3}},
    keywordstyle=[2]{\ttfamily\color{Green4}},
    keywordstyle=[3]{\ttfamily\color{orange}},
    keywordstyle=[4]{\ttfamily\color{violet}},
    keywordstyle=[5]{\ttfamily\color{red}}, 
    keywords = {cmova, cmovae, cmovb, cmove, cmovne, cmovbe, mov, xor, or,  jbe, jne, add, jmp,  cmp, shl, and, sar, lfence, jae, clt,  and, lea},
    morekeywords = {},
    morekeywords = [2]{A, B},
    morekeywords = [3]{\%rax, \%rbx, \%rcx, \%eax, \%ecx, \%rbp, \%cl, \%rdi, \%edx, \%esi, \%rsi, \%rdx, \%rsp, \%al, \%rip, \%ax, \%bx, \%cx,  \%bp,  \%di, \%dx, \%si, \%sp, \%ip, \%r8b},
    morekeywords = [4]{k, y, size, temp},
    morekeywords = [5]{END, .L1, .L2}
}

\newcommand{\inlineCcode}[1]{\lstinline[style=Cstyle]|#1|}
\newcommand{\inlineASMcode}[1]{\lstinline[style=ASMstyle]|#1|}

%%%%%%%% NOTES
\definecolor{Blue3}{HTML}{0000CD}
\definecolor{Green4}{HTML}{008B00}
\definecolor{Red3}{HTML}{CD0000}
\definecolor{orange}{rgb}{0.8, 0.47, 0.196}



\newcommand\boris[1]{\textcolor{Blue3}{Boris: #1}}
\newcommand\marco[1]{\textcolor{Red3}{Marco: #1}}
\newcommand\jan[1]{\textcolor{Green4}{Jan: #1}}
\newcommand\jose[1]{\textcolor{brown}{Jos{\'e}: #1}}
\newcommand\andres[1]{\textcolor{violet}{Andr{\'e}s: #1}}


\newcommand\remove[1]{}

%%%%%%%% NEW SYMBOLS
\newcommand{\cL}{{\cal L}}
\renewcommand{\leadsto}{\rightsquigarrow}
 \newcommand\xleadsto[1]{%
   \if\relax\detokenize{#1}\relax
   \rightsquigarrow
   \else
   \mathrel{%
     \begin{tikzpicture}[%
       baseline={(current bounding box.south)}
       ]
       \node[%
       ,inner sep=.44ex
       ,align=center
       ] (tmp) {$\scriptstyle #1$};
       \path[%
       ,draw,<-
       ,decorate,decoration={%
         ,zigzag
         ,amplitude=0.7pt
         ,segment length=1.2mm,pre length=3.5pt
       }
       ] 
       (tmp.south east) -- (tmp.south west);
     \end{tikzpicture}
   }
   \fi
 }


\newcommand{\todo}[1]{{\color{red}[Todo: #1]}}
\newcommand{\comments}[1]{{\color{blue}[Comment: #1]}}


%%%%% LANGUAGE

\newcommand{\histories}{\mathcal{H}}
\newcommand{\tup}[1]{\langle #1 \rangle}
\newcommand{\Var}{\mathit{Regs}}
\newcommand{\Val}{\mathit{Vals}}
\newcommand{\Nat}{\mathbb{N}}
\newcommand{\Lbl}{\mathit{Labels}}
\newcommand{\Mem}{\mathit{Mem}}
\newcommand{\Assgn}{\mathit{Assgn}}
\newcommand{\Addr}{\mathit{Addrs}}
\newcommand{\Conf}{\mathit{Conf}}
\newcommand{\Init}{\mathit{InitConf}}
\newcommand{\Final}{\mathit{FinalConf}}
\newcommand{\InitAssgn}{\mathit{InitAssgn}}
\newcommand{\FinalAssgn}{\mathit{FinalAssgn}}
\newcommand{\Obs}{\mathit{Obs}}
\newcommand{\ExtObs}{\mathit{ExtObs}}
\newcommand{\Prg}{\mathit{Prg}}
\newcommand{\bp}{{\mathcal{O}}}
\newcommand{\lbl}{\ell}
\newcommand{\kywd}[1]{\mathbf{#1}}
\newcommand{\skipKywd}{\kywd{skip}}
\newcommand{\storeKywd}{\kywd{store}}
\newcommand{\loadKywd}{\kywd{load}}
\newcommand{\jmpKywd}{\kywd{jmp}}
\newcommand{\jzKywd}{\kywd{beqz}}
\newcommand{\pcKywd}{\kywd{pc}}
\newcommand{\barrierKywd}{\kywd{spbarr}}
\newcommand{\pc}{\pcKywd}
\newcommand{\obsKywd}[1]{\textcolor{Blue3}{\mathtt{#1}}}
\newcommand{\loadObsKywd}{\obsKywd{load}}
\newcommand{\storeObsKywd}{\obsKywd{store}}
\newcommand{\pcObsKywd}{\obsKywd{pc}}
\newcommand{\loadObs}[1]{\loadObsKywd\ #1}
\newcommand{\storeObs}[1]{\storeObsKywd\ #1}
\newcommand{\pcObs}[1]{\pcObsKywd\ #1}
\newcommand{\startObsKywd}[1]{\obsKywd{start}}
\newcommand{\commitObsKywd}[1]{\obsKywd{commit}} 
\newcommand{\rollbackObsKywd}[1]{\obsKywd{rollback}}
\newcommand{\startObs}[1]{\startObsKywd{}\ #1}
\newcommand{\finishObs}{\kywd{end}}
\newcommand{\commitObs}[1]{\commitObsKywd{}\ #1}
\newcommand{\rollbackObs}[1]{\rollbackObsKywd{}\ #1}
\newcommand{\unaryOp}[1]{\ominus #1}
\newcommand{\binaryOp}[2]{#1 \otimes #2}
\newcommand{\cmd}[2]{ #1 : #2 }
\newcommand{\pseq}[2]{#1 ; #2}
\newcommand{\pskip}{\skipKywd{}}
\newcommand{\passign}[2]{#1 \leftarrow #2}
\newcommand{\pload}[2]{\loadKywd\ #1, #2}
\newcommand{\pstore}[2]{\storeKywd\ #1, #2}
\newcommand{\pcondassign}[3]{#1 \xleftarrow{#3?} #2}
\newcommand{\pjmp}[1]{\jmpKywd\ #1}
\newcommand{\pjz}[2]{\jzKywd\ #1, #2}
\newcommand{\pbarrier}{\barrierKywd}
\newcommand{\select}[2]{#1(#2)}
\newcommand{\eval}[2]{\xrightarrow{#2}}
\newcommand{\speval}[3]{\xleadsto{#3}}
\newcommand{\exprEval}[2]{\llbracket #1 \rrbracket(#2)}
\newcommand{\isJump}[2]{\mathit{isJmp}(#1,#2)}
\newcommand{\traces}[2]{\llparenthesis #1 \rrparenthesis_{#2}}
\newcommand{\project}[1]{#1\mathord{\upharpoonright}}
\newcommand{\specProject}[1]{#1\mathord{\upharpoonright_{se}}}
\newcommand{\nspecProject}[1]{#1\mathord{\upharpoonright_{nse}}}
\newcommand{\filter}[2]{\mathit{filter}(#1,#2)}
\newcommand{\access}[1]{\mathit{acc}(#1)}
\newcommand{\specStates}{\mathit{SpecS}}
\newcommand{\ExtConf}{\mathit{ExtConf}}
\newcommand{\exhausted}[1]{\mathit{enabled}(#1)}
\newcommand{\epochEnd}[2]{\mathit{next}(#1,#2)}
\newcommand{\sw}{\mathit{sw}}
\newcommand{\deriveObs}[2]{\mathit{obs}(#1,#2)}
\newcommand{\misprediction}[2]{\mathit{mispred}(#1,#2)}
\newcommand{\zeroes}[1]{\mathit{zeroes}(#1)}
\newcommand{\zeros}[1]{\zeroes{#1}}
\newcommand{\topzeroes}[1]{\mathit{ltz}(#1)}
\newcommand{\id}{\mathit{id}}
\newcommand{\ctr}{\mathit{ctr}}
\newcommand{\committed}[1]{\mathit{cmt}(#1)}
\newcommand{\rolledback}[1]{\mathit{rb}(#1)}
\newcommand{\filterComm}[1]{\mathit{filter}_C(#1)}
\newcommand{\filterRoll}[1]{\mathit{filter}_R(#1)}
\newcommand{\eqvclass}[1]{\llbracket #1 \rrbracket}
\newcommand{\eqvclassNsteps}[2]{\llbracket #1 \rrbracket_{#2}}
\newcommand{\reachable}[2]{\mathit{reachable}_{#2}(#1)}
\newcommand{\expand}[2]{\textsc{expand}(#1,#2)}
\newcommand{\rbTrans}[1]{\mathit{rbTrans}(#1)}
\newcommand{\maxRbTrans}[1]{\mathit{maxRbTrans}(#1)}
\newcommand{\rbTransDepth}[2]{\mathit{rbTrans}(#1,#2)}
\newcommand{\atLeast}[2]{\mathit{atLeast}(#1,#2)}
\newcommand{\topId}[1]{\mathit{topId}(#1)}
\newcommand{\depth}[2]{\mathit{depth}(#1,#2)}
\newcommand{\trace}[1]{\mathit{trace}(#1)}
\newcommand{\init}[1]{\mathit{init}(#1)}
\newcommand{\final}[1]{\mathit{final}(#1)}
\newcommand{\maxDepth}[1]{\mathit{maxDepth}(#1)}
\newcommand{\maxWindow}[1]{\mathit{maxWind}(#1)}
\newcommand{\conf}[1]{\mathit{conf}(#1)}
\newcommand{\ec}{\rho}
\newcommand{\trans}[2]{\mathit{transEx}(#1,#2)}
\newcommand{\transNr}[1]{\mathit{transNr}(#1)}
\newcommand{\maxTransSet}[1]{\mathit{maxTrans}(#1)}
\newcommand{\allTransSet}[1]{\mathit{allTrans}(#1)}
\newcommand{\body}[1]{\mathit{body}(#1)}
\newcommand{\pcSim}{\sim_{\pc}}
\newcommand{\nextSim}{\sim_{\mathit{next}}}
\newcommand{\steps}[1]{\mathit{steps}(#1)}
\newcommand{\mbp}{{\mathcal{O}}}


%%%%% SEQUENCES 
\newcommand{\finiteSequences}[1]{#1^*}
\newcommand{\length}[1]{|#1|}
\newcommand{\prefix}[2]{{#1}^{#2}}
\newcommand{\elt}[2]{#1|_{#2}}
\newcommand{\emptysequence}{\varepsilon}
\newcommand{\concat}{\cdot}
\newcommand{\decrement}[1]{\mathit{decr}(#1)}

%%%%%% SETS
\newcommand{\powerset}[1]{2^{#1}}

%%%%% SEC. CONDITIONS
\newcommand{\knowledge}[4]{K_{#1}(#2,#3, #4)}
\newcommand{\indist}[1]{\sim_{#1} }


%%%%% SYMBOLIC SEMANTICS
\newcommand{\sexpr}{\mathit{se}}
\newcommand{\mem}[1]{\mathbf{mem}(#1)}
\newcommand{\SymVal}{\mathit{SymbVals}}
\newcommand{\SymExpr}{\mathit{SymbExprs}}
\newcommand{\ite}[3]{\mathbf{ite}(#1,#2,#3)}
\newcommand{\encodeLoad}[2]{\mathit{encodeLoad}(#1,#2)}
\newcommand{\symEval}[2]{\xrightarrow{#2}_{\mathit{s}}}
\newcommand{\symExprEval}[2]{#2(#1)}
\newcommand{\SymObs}{\mathit{SymObs}}
\newcommand{\SymExtObs}{\mathit{SymExtObs}}
\newcommand{\SymConf}{\mathit{SymConf}}
\newcommand{\SymExtConf}{\mathit{SymExtConf}}
\newcommand{\symTraces}[2]{\mathit{symTraces}_{#1}(#2)}
\newcommand{\symEvalTraces}[1]{\mathit{symTraces}_{\rightarrow}(#1)}
\newcommand{\symSpevalTraces}[1]{\mathit{symTraces}_{\xleadsto{}}(#1)}
\newcommand{\concreteJump}[2]{\mathit{concrJump}(#1,#2)}
\newcommand{\symbolicJump}[2]{\mathit{symbJump}(#1,#2)}
\newcommand{\otherPc}[3]{\mathit{mispred}(#1,#2,#3)}
\newcommand{\symb}[1]{#1_s}
\newcommand{\symWrite}[3]{\kywd{write}(#1,#2,#3)}
\newcommand{\symRead}[2]{\kywd{read}(#1,#2)}
\newcommand{\symModels}[1]{\mathit{models}(#1)}
\newcommand{\determines}[3]{\mathit{determines}_{{#3}}(#1,#2)}
\newcommand{\low}{\kywd{LOW}}
\newcommand{\corresponds}[3]{\mathit{corrTraces}_{#2,#3}(#1)}
\newcommand{\policy}{P}
\newcommand{\rename}[1]{\mathit{rename}(#1)}
\newcommand{\symPcObs}[1]{\obsKywd{symPc}(#1)}
\newcommand{\cstrs}[1]{\mathit{obsEqv}(#1)}
\newcommand{\pathCond}[1]{\mathit{pthCnd}(#1)}
\newcommand{\policyEqv}[1]{\mathit{polEqv}(#1)}
\newcommand{\memcheck}{\textsc{MemLeak}}
\newcommand{\pccheck}{\textsc{CtrlLeak}}
\renewcommand{\algorithmicrequire}{\textbf{Input:}}
\renewcommand{\algorithmicensure}{\textbf{Output:}}
\algdef{SE}[DOWHILE]{Do}{doWhile}{\algorithmicdo}[1]{\algorithmicwhile\ #1}%
\newcommand{\sameSymPc}[1]{\mathit{sameSymbPc}(#1)}
\newcommand{\decrTop}[1]{\mathit{decr}'(#1)}
\newcommand{\decrementTop}[1]{\decrTop{#1}}
\newcommand{\window}[1]{\mathit{wndw}(#1)} 
\newcommand{\exhaustedTop}[1]{\mathit{enabled}'(#1)}
\newcommand{\zeroesTop}[1]{\mathit{zeroes}'(#1)}


%%%%% CASE STUDIES
\newcommand{\unp}{\textsc{Unp}}
\newcommand{\fen}{\textsc{Fen}}
\newcommand{\slh}{\textsc{Slh}}
\newcommand{\opt}{\texttt{-O2}}
\newcommand{\unopt}{\texttt{-O0}}

%%%%% NOTATION FOR PROGRAM EXECUTION
\newcommand{\lbbar}{\{\kern-0.5ex|}
\newcommand{\rbbar}{|\kern-0.5ex\}}
\newcommand{\lanbar}{\langle \kern-0.5ex|}
\newcommand{\ranbar}{|\kern-0.5ex\rangle}
\newcommand{\specEval}[3]{\llbracket #1 \rrbracket_{#2}(#3)}
\newcommand{\nspecEval}[2]{\llparenthesis #1 \rrparenthesis(#2)}
\newcommand{\amEval}[3]{\lbbar #1 \rbbar_{#2}(#3)}
\newcommand{\ameval}[1]{\xLongrightarrow{#1}}

\newcommand{\symSpeval}[3]{\xLongrightarrow{#3}_{\mathit{s}}}

\newcommand{\nspecTraces}[1]{\llparenthesis #1 \rrparenthesis}
\newcommand{\specTraces}[2]{\llbracket #1 \rrbracket_{#2}}
\newcommand{\amTraces}[2]{\lbbar #1 \rbbar_{#2}}
\newcommand{\symbTraces}[2]{\amTraces{#1}{#2}^\mathit{symb}} 